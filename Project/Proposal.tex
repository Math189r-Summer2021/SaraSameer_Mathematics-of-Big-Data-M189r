\documentclass{article}
\usepackage[utf8]{inputenc}
\usepackage{hyperref}

\title {MATH189R Course Project Proposal \\
\large Title: Cervical Cancer Risk Analysis Through ML Algorithms  }
\author{Submitted by Sara Sameer }
\date{ 01 July 2021}

\begin{document}

\maketitle

\section*{Project Description}
\smallskip
Cervical cancer is a type of cancer that occurs in the cells of the cervix — the lower part of the uterus that connects to the vagina. Before the condition of cancer gets worsen, it is crucial to identify this problem and deal with its causes.Under-privileged areas, such as Pakistan, has lack of medical resources and expensive medical procedures. Most of the population is deprived of the screening test facility. Thus, I want to work on this project to present a productive approach towards testing the risk of cervical cancer among people.\\\\
\smallskip
In my project, I will analyse the likelihood of having cervical cancer by determining its predictors through Machine Learning algorithms like Naive Bayes, Regression for prediction, Decision trees, Support Vector Machine etc. I aim to implement an algorithm that provides most accurate output to cope with this deadly disease.

\section*{Resources}

\large Open Source Codes:\\
\url{https://github.com/Charleswow/Cervical-Cancer-Risk-Detection-Project}
\url{https://github.com/lauramann/cervicalCancerAnalysis}\\\\
\smallskip
\large Coursera Guided Project\\
\url{https://www.coursera.org/projects/cervical-cancer-risk-prediction-using-machine-learning}\\\\
\smallskip
\large Research Papers \\
\url{https://www.ncbi.nlm.nih.gov/pmc/articles/PMC7416093/} \\
\url{https://www.ripublication.com/ijaer19/ijaerv14n11_07.pdf}\\




\end{document}
